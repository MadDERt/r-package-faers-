\documentclass{bioinfo}
\copyrightyear{2025} \pubyear{2025}



% tightlist command for lists without linebreak
\providecommand{\tightlist}{%
  \setlength{\itemsep}{0pt}\setlength{\parskip}{0pt}}




% hyperref makes the margins screwy.
% https://groups.google.com/forum/#!topic/latexusersgroup/4W_SwGk6zx4
% http://ansuz.sooke.bc.ca/software/latex-tricks.php
% \usepackage[colorlinks=true, allcolors=blue]{hyperref}

\access{Advance Access Publication Date:   }
\appnotes{Application Note}

\begin{document}


\firstpage{1}

\subtitle{Genome analysis}

\title[Bioinformatics Rmd Template]{faers: An R Package for Seamlessly
Bridging the FAERS Database to R and Delivering Unified
Pharmacovigilance Workflows}

\author[FirstAuthorLastName \textit{et~al}.]{
Shixiang Wang\,\textsuperscript{1}, Yun
Peng\,\textsuperscript{2}, Zhangyu Wang\,\textsuperscript{1}
}

\address{
\textsuperscript{1}Central South University\\
\textsuperscript{2}aaa\\
}

\corresp{To whom correspondence should be addressed. E-mail: qq}

\history{Received on XXX; revised on XXX; accepted on XXX}

\editor{Associate Editor: XXX}

\abstract{
\textbf{Motivation:} The FDA Adverse Event Reporting System (FAERS)
serves as a critical database for monitoring post-marketing drug
safety.However, the primary focus on safety signal detection within
FAERS has left a significant gap in integrating pharmacovigilance
analysis with genetic tools.Thus, we aim to effectively utilize FAERS
data to bridge pharmacovigilance with genetic analysis, thereby
enhancing the precision of safety assessments and facilitating the
development of personalized medicine approaches.\\
\textbf{Results:} We developed the R package faers, which seamlessly
connects the FAERS database to the R programming language and provides a
unified approach for the seamless execution of pharmacovigilance
analysis and the integration of genetic tools in R.\\
\textbf{Availability:} faers is available on CRAN and on GitHub at
(https://github.com/Yunuuuu/faers).\\
\textbf{Contact:}212@qq.com\\
\textbf{Supplementary information:} Supplementary data are available at
Bioinformatics Online.}

\maketitle

\section{Introduction}

The FDA Adverse Event Reporting System (FAERS) is the core database for
post-marketing safety monitoring of all approved drugs and therapeutic
biologics by the FDA. As required by regulations, it consolidates
mandatory adverse event reports submitted by pharmaceutical companies,
as well as voluntary first-hand information reported by individuals and
healthcare professionals. The data, which has been collected since
January 2004 and is continuously updated on a quarterly basis, includes
eight types of documents: demographic and administrative information,
detailed drug information, indications for use, report sources, drug
start and end dates, patient outcomes, reports of therapeutic failure,
and adverse events. Its core value is reflected in key areas such as
early safety signal detection, risk assessment, and drug labeling
revisions, providing direct evidence for the FDA to formulate drug
safety policies and risk management measures.

Currently, FAERS research has formed a tripartite landscape of
``official---open-source---R ecosystem'': The official path still
requires researchers to download quarterly files, deduplicate and clean
them, recode with MedDRA, and then manually invoke algorithms. Although
the FAERS Public Dashboard provides zero-code visualization and
download, it lacks a batch analysis interface. In the open-source
platform, OpenVigil 2.1, by pre-hosting cleaned data from 2004 to 2022,
supports one-click output of PRR/ROR after online setting of
drug-event-time window, but its cross-platform reproducibility is
limited. AERSMine, leveraging Spark to integrate 2.1 million reports,
offers interactive differential efficacy forest plots, yet still cannot
directly connect to the Bioconductor genetics pipeline. DiAna\_package
encapsulates downloading, cleaning, standardization, signaling, and
genetics integration into a single R command, being the first to
incorporate population genotypes into the signal model, but it leaves
gaps in environment locking and full-field integrity, which provides an
entry point for the proposed unified faers framework in this paper.

To systematically address the shortcomings of existing tools in terms of
reproducibility, integration into the R ecosystem, and in-depth mining
of genetic information, we developed the faers package. This package
connects the FAERS database to R through a unified pipeline, enabling
one-click data download, cleaning, standardization, and signal
detection. It also seamlessly integrates with the Bioconductor genetics
analysis framework, supporting high-throughput association analysis of
population-differentiated polymorphisms and drug pathways, thereby
providing reproducible and scalable decision support for precision
medicine.

\section{Methods}

The workflow for pharmacovigilance analysis using the faers package
mainly consists of the following four steps: (i) check metadata of
FAERS; (ii) downloading and parsing the quarterly FAERS data files;
(iii) standardization and deduplication; (iv) conducting
pharmacovigilance analysis.

\subsection{Metadata Inspection}

We obtained the FAERS quarterly index with faers\_meta(), which returns
a data.table listing year, quarter, period, file URLs and sizes
(ASCII/XML, 50--140 MB/quarter) for all releases since 2004Q1. By
default (force = FALSE, internal = !curl::has\_internet()), the function
first checks a local cache; if absent, it parses the latest index from
the FDA site
(https://fis.fda.gov/extensions/FPD-QDE-FAERS/FPD-QDE-FAERS.html),
ensuring reproducibility both online and offline.

\subsection{Download and Parse quarterly data files from FAERS}

Quarterly files are raw data extracted from the FAERS database,
typically stored in ASCII or SGML format, and record a series of
information related to drug adverse reactions for a specific period,
including: demo:demographic and administrative details; drug:suspect and
concomitant drugs; reac:adverse reactions coded to MedDRA; outc:patient
outcomes; rpsr:reporter source; ther:drug-therapy dates;
indi:indications (MedDRA-coded diagnoses).

Data acquisition was performed with faers(years, quarters, dir,
compress\_dir).years and quarters specify the desired quarters, while
dir and compress\_dir indicate the raw and extracted file locations.
Internally, faers() calls faers\_download() and faers\_parse() to
preprocess each ASCII file (default format) from the FAERS repository,
returning a FAERSascii object; multiple quarters are combined via
faers\_combine(). After download, faers\_get() retrieves any field as a
data.table.

\subsection{Standardize and De-duplication}

Each quarterly FAERS file stores adverse reactions (reac) and drug
indications (indi) as MedDRA Preferred Terms (PT). To ensure
terminological consistency, we first standardise these fields with
faers\_standardize(data, meddra\_path), where meddra\_path points to the
locally unzipped MedDRA ASCII directory. This step embeds the complete
MedDRA dictionary (hierarchy + SMQ data) in the meddra slot of the
returned object. The optional use argument in faers\_meddra() allows
selective retrieval of either component. As a by-product, three columns
are automatically appended to indi and reac: meddra\_hierarchy\_idx,
meddra\_hierarchy\_from, meddra\_code \& meddra\_pt. These columns are
generated by faers\_meddra() and are immediately available via
faers\_get(``indi'') or faers\_get(``reac''), eliminating manual
mapping. Finally, to address FAERS' inherent duplication and
incompleteness, we treat two reports as duplicates if they match on
drugs and reactions but differ in at most one of: sex, age, reporting
country, event date, start date or drug indications. Deduplication is
performed with faers\_dedup(), yielding a clean, case-level dataset
ready for downstream analysis.

\subsection{Pharmacovigilance analysis}

Details for Method 2. Lorem ipsum ad nauseum. Introduce your topic.
Lorem ipsum ad nauseum. Introduce your topic. Lorem ipsum ad nauseum.
Introduce your topic. Lorem ipsum ad nauseum. Introduce your topic.
Lorem ipsum ad nauseum.

Details for Method 2. Lorem ipsum ad nauseum. Introduce your topic.
Lorem ipsum ad nauseum. Introduce your topic. Lorem ipsum ad nauseum.
Introduce your topic. Lorem ipsum ad nauseum. Introduce your topic.
Lorem ipsum ad nauseum.

\section{Discussion}

Discussion of results. Lorem ipsum ad nauseum. Introduce your topic.
Lorem ipsum ad nauseum. Introduce your topic. Lorem ipsum ad nauseum.
Introduce your topic. Lorem ipsum ad nauseum. Introduce your topic.
Lorem ipsum ad nauseum.

Discussion of results. Lorem ipsum ad nauseum. Introduce your topic.
Lorem ipsum ad nauseum. Introduce your topic. Lorem ipsum ad nauseum.
Introduce your topic. Lorem ipsum ad nauseum. Introduce your topic.
Lorem ipsum ad nauseum.

Discussion of results. Lorem ipsum ad nauseum. Introduce your topic.
Lorem ipsum ad nauseum. Introduce your topic. Lorem ipsum ad nauseum.
Introduce your topic. Lorem ipsum ad nauseum. Introduce your topic.
Lorem ipsum ad nauseum.

Discussion of results. Lorem ipsum ad nauseum. Introduce your topic.
Lorem ipsum ad nauseum. Introduce your topic. Lorem ipsum ad nauseum.
Introduce your topic. Lorem ipsum ad nauseum. Introduce your topic.
Lorem ipsum ad nauseum.

\section{Conclusion}

Anything else? Lorem ipsum ad nauseum. Introduce your topic. Lorem ipsum
ad nauseum. Introduce your topic. Lorem ipsum ad nauseum. Introduce your
topic. Lorem ipsum ad nauseum. Introduce your topic. Lorem ipsum ad
nauseum.

Anything else? Lorem ipsum ad nauseum. Introduce your topic. Lorem ipsum
ad nauseum. Introduce your topic. Lorem ipsum ad nauseum. Introduce your
topic. Lorem ipsum ad nauseum. Introduce your topic. Lorem ipsum ad
nauseum.

Anything else? Lorem ipsum ad nauseum. Introduce your topic. Lorem ipsum
ad nauseum. Introduce your topic. Lorem ipsum ad nauseum. Introduce your
topic. Lorem ipsum ad nauseum. Introduce your topic. Lorem ipsum ad
nauseum.

Anything else? Lorem ipsum ad nauseum. Introduce your topic. Lorem ipsum
ad nauseum. Introduce your topic. Lorem ipsum ad nauseum. Introduce your
topic. Lorem ipsum ad nauseum. Introduce your topic. Lorem ipsum ad
nauseum.

\section*{Acknowledgements}
\addcontentsline{toc}{section}{Acknowledgements}

These should be included at the end of the text and not in footnotes.
Please ensure you acknowledge all sources of funding, see funding
section below.

Details of all funding sources for the work in question should be given
in a separate section entitled `Funding'. This should appear before the
`Acknowledgements' section.

\section*{Funding}
\addcontentsline{toc}{section}{Funding}

The following rules should be followed:

\begin{itemize}
\tightlist
\item
  The sentence should begin: `This work was supported by \ldots{}' -
\item
  The full official funding agency name should be given, i.e.~`National
  Institutes of Health', not `NIH' (full RIN-approved list of UK funding
  agencies)
\item
  Grant numbers should be given in brackets as follows: `{[}grant number
  xxxx{]}'
\item
  Multiple grant numbers should be separated by a comma as follows:
  `{[}grant numbers xxxx, yyyy{]}'
\item
  Agencies should be separated by a semi-colon (plus `and' before the
  last funding agency)
\item
  Where individuals need to be specified for certain sources of funding
  the following text should be added after the relevant agency or grant
  number `to {[}author initials{]}'.
\end{itemize}

An example is given here: `This work was supported by the National
Institutes of Health {[}AA123456 to C.S., BB765432 to M.H.{]}; and the
Alcohol \& Education Research Council {[}hfygr667789{]}.'

Oxford Journals will deposit all NIH-funded articles in PubMed Central.
See Depositing articles in repositories -- information for authors for
details. Authors must ensure that manuscripts are clearly indicated as
NIH-funded using the guidelines above.


% Bibliography
\bibliographystyle{natbib}
\bibliography{bibliography.bib}

\end{document}
